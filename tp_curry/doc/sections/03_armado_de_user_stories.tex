\section{Armado de User Stories}

Para este proceso lo que hicimos fue armar un conjunto de User Stories que, al resolverlas, consideremos como finalizado el trabajo.
En ese sentido, probablemente elegiríamos pocas stories pero muy generales y que por ende serían necesario subdividir en subitems.

Luego de esto definimos el esfuerzo y business value de cada una para poder armar el primer sprint.

Veamos detalles y decisiones de cada una de estas partes.

\subsection{Actores}

Las user stories que pensábamos íban dirigidas a dos tipos de actores:

\begin{itemize}

\item Diseñador de Juego (DJ): Representa al integrante del Stakeholder que modificará el comportamiento configurable de la aplicación. Por ejemplo, existe un cap inicial para cada participante, que en un futuro podría cambiar y que debería ser algo manipulable por este actor.

\item Participante : representa a la persona que usará la aplicación en producción.

\end{itemize}


\subsection{Asignación de esfuerzo}

El proceso fue recorrer las User Stories de comienzo a fin y en cada paso asignamos un valor, siempre comparando con los anteriores. 

Esto se contrapuso a la propuesta de la materia que planteaba leer todas primero, definir una que represente a la unidad de Esfuerzo y en base a eso se define el resto.

En algunos casos la forma en la que trabajamos nos obligó a volver hacia atrás a recalcular el valor de esfuerzo que habíamos asignado a algunas de las stories, en función de cosas que surgían de stories posteriores.

